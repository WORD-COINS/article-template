\makeatletter
\def\input@path{{../../}}
\makeatother
 
\documentclass[../../main]{subfiles}

% ここに\usepackageなどを書いても無意味です。
% リポジトリ直下のmain.texに書きましょう。

\begin{document}

\lstset{
  basicstyle=\ttfamily,
  commentstyle=\textit,
  frame=trBL,
  numbers=left,
  breaklines=true,
  title=\lstname,
}

\definecolor{diffstart}{named}{Grey}
\definecolor{diffincl}{named}{Green}
\definecolor{diffrem}{named}{OrangeRed}

\lstdefinelanguage{diff}{
  morecomment=[f][\color{diffstart}]{@@},
  morecomment=[f][\color{diffincl}]{+\ },
  morecomment=[f][\color{diffrem}]{-\ },
}

% Listingsの番号再定義です。
% この命令はdocumentがはじまってからでないとダメなのでここにあります。
\renewcommand{\thelstlisting}{\arabic{lstlisting}}

\subtitle{ヘッダの見出し}

% 著者名の前にある「文 編集部」を削除する場合は次のコマンドをアンコメントします。
% \authormark{}

\author{情報 太郎}

\chapter{記事を執筆しよう}

\section{環境を準備する}

WORDの記事をコンパイルするにあたって、\TeX Liveといった
\TeX のツールセットが必要です。

\begin{itemize}
  \item Git\footnote{\url{https://git-scm.com/}}
  \item \TeX Live\footnote{\url{https://www.tug.org/texlive/}}
  またはMac\TeX\footnote{\url{http://www.tug.org/mactex/}}
  \item GNU Make\footnote{\url{https://www.gnu.org/software/make/}}
\end{itemize}

\section{コンパイルする}
\label{sec:compile}

記事をコンパイルするため、記事のGitリポジトリのルートで次のコマンドを実行します。

\begin{lstlisting}
$ git submobule update --init
$ make
\end{lstlisting}

\ovalbox{main.pdf}が作成されれば成功です。
ただし、\lstinline|git submobule update --init|は最初のみ必要で、
その後は\lstinline|make|のみでよいです。

\section{記事を追加する}

まず、\ovalbox{./articles/hinagata}というディレクトリをコピーして任意の名前で
\ovalbox{articles}ディレクトリへ置きます。
そして、Gitのリポジトリルートにある\ovalbox{main.tex}を開きます。

\begin{lstlisting}[language=TeX]
\mainmatter

% 記事(サンプル)
\article{./articles/hinagata}

% 裏表紙
\article{./articles/back_cover}
\end{lstlisting}

このようになっていると思いますので、この「記事(サンプル)」と「裏表紙」という
コメントの間に次のような書き込みをしましょう。

\begin{lstlisting}[language=TeX]
\article{./articles/my_article_directory_name}
\end{lstlisting}

そして\ref{sec:compile}節に従って\lstinline|make|を実行し、
記事が適切にコンパイルされることを確認します。

また、次のような表示でコンパイルが実行されない場合があります。

\begin{lstlisting}
make: Nothing to be done for `all'.
\end{lstlisting}

このような場合はリポジトリルートの\ovalbox{main.pdf}を削除して
\lstinline|make|を再び実行します。

\section{Gitサーバにpushする}

WORDではかつて伝統あるGitoliteにより管理されていましたが、
ディスク故障によってGitのデータを失ってからBitbucketへ移行しました。

まず、WORDのGitbucketのチームである\url{https://bitbucket.org/word-coins}へ
参加していない場合は、Bitbucketのアカウントを取得して
管理者の誰かへ連絡してメンバーに加えてもらいましょう。

Bitbukectにプッシュする場合は、自分の編集を行った適当なブランチを
次のコマンドでプッシュします。

\begin{lstlisting}[mathescape]
git push origin feature/my_branch
\end{lstlisting}

プッシュを成功させた場合には、ビルドの結果がSlack\footnote{\url{https://word-ac.slack.com}}の\texttt{\#latex\_feed}チャンネルに流れます。

\section{トラブルシューティング}

\subsection{「文 編集部」の削除}

編集部以外のメンバーが執筆する場合「文 編集部」は必要ありません。
「文 編集部」は以下のコマンドを\lstinline|\documentclass|から
\lstinline|\begin{document}|の間のどこかに書くことで消せます。

\begin{lstlisting}[language=TeX, mathescape]
\authormark{}    
\end{lstlisting}

\subsection{\texttt{\textbackslash input}コマンド}

\lstinline|\input|コマンドを利用する場合、
このフォルダからの相対パスではなくリポジトリルートからの相対パスを利用してください。

\subsection{CIが通らない}

BitbucketのCI環境ではDocker\footnote{\url{https://www.docker.com/}}を利用して
コンパイルを実行しています。もしCI環境でのみコンパイルに失敗する場合は、
上記のDockerと\lstinline|docker-compose|\footnote{\url{https://github.com/docker/compose}}を
利用して次のようにコンパイルができるか試すのがよいでしょう。

\begin{lstlisting}
$ docker-compose pull
$ docker-compose up
\end{lstlisting}

もしコンパイルに失敗する場合はパッケージが不足していると考えられます。
\ovalbox{./docker/Dockerfile}を編集してパッケージを追加しましょう。

\section{他の問題について}

問題があればSlackの\texttt{\#latex}チャンネルや、編集会議で聞くと良いでしょう。

直接詳しい人にSNSで聞く場合、@\_yyu\_\footnote{\url{https://twitter.com/_yyu_}}へ投げると早い。
Lua\LaTeX に関しては@Nymphium\footnote{\url{https://twitter.com/Nymphium}}か@azuma962\footnote{\url{https://twitter.com/azuma962}}へ。
クラスファイルの全体的な質問は@hid\_alma1026\footnote{\url{https://twitter.com/hid_alma1026}}へ。

\end{document}
