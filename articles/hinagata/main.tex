\usepackage{listings}
\lstset{
  basicstyle=\ttfamily,
  commentstyle=\textit,
  frame=trBL,
  numbers=left,
  breaklines=true,
  title=\lstname,
}
\usepackage{fancybox}
\usepackage{url}
\usepackage{hyperref}
\ifluatex
\else
  \usepackage{pxjahyper}
\fi

\subtitle{ヘッダの見出し}

% 著者名の前にある「文 編集部」を削除する場合は次のコマンドをアンコメントします。
% \authormark{}

\author{ほげ}

\begin{document}

\chapter{記事の書き方}

\section{まずはじめに}

\subsection{p\LaTeX を使う}

\subsubsection{macOS・Linux}

$article\_name$は適当な名前として、以下のようなコマンドでブランチを分けましょう。

\begin{lstlisting}[mathescape]
git submodule update --init
git checkout -b personal/$username$/$article\_name$
cd ./articles
cp -r ./hinagata ./my-article-name
cd ./my-article-name
make
\end{lstlisting}

\subsubsection{Windows}

WORDクラスファイルはWindowsでもコンパイルすることができます。
次のようにコマンドを実行します。

\begin{lstlisting}[mathescape]
git submodule update --init
git checkout -b personal/$username$/$article\_name$
cd ./articles
cp -r ./hinagata ./my-article-name
cd ./my-article-name
make
\end{lstlisting}

\subsection{Lua\LaTeX を使う}

WORDでは新たにLua\LaTeX が使えるようになりました。

\begin{description}
  \item[macOS・Linux] \lstinline|make|のかわりに\lstinline|LATEXMKFLAG=-lualatex make|を利用する。
 
  \item[Windows] \lstinline|make|の前に\lstinline|set LATEXMKFLAG=-lualatex|として、環境変数を設定する。
\end{description}

\section{記事を書く}

記事を書いたら、\lstinline|make|コマンドでビルドできます。

\begin{lstlisting}
git add *
make
\end{lstlisting}

これで\ovalbox{main\_odd.pdf}、\ovalbox{main\_even.pdf}が生成されれば成功です。
あとは\ovalbox{main.tex}を編集すれば記事が出来ます。

\section{Gitサーバにpushする}

記事のキリの良いところで\lstinline|git push|するといいのですが、最初のpushの時には、
origin\footnote{ここではWORDのGitサーバである\url{gitolite.word-ac.net}のことです}%
に新しいブランチを登録する必要があります。それは以下のようにしましょう。

\begin{lstlisting}[mathescape]
git push origin personal/$username$/$article\_name$
\end{lstlisting}

pushを成功させた場合には、ビルドの結果がslack\footnote{\url{https://word-ac.slack.com}}の\#jenkinsチャンネルに流れます。
slackを見ていない場合は、\url{https://jenkins.word-ac.net/job/LaTeX/}および\url{https://gitiles.word-ac.net/}を見ると良いでしょう。

\section{ヒラギノフォントを埋め込む}

macOSを利用しているなど、手元のコンパイル環境でヒラギノフォントが利用可能な場合は、
次の手順でヒラギノフォントを埋め込んだPDFファイルを作成できます。

\subsection{p\LaTeX の場合}

\begin{lstlisting}
sudo cjk-gs-integrate --link-texmf --force
sudo mktexlsr  
sudo kanji-config-updmap-sys hiragino-elcapitan-pron
\end{lstlisting}
  
この状態で\lstinline|make|することでヒラギノフォント埋め込みPDFが作成されます。

\subsection{Lua\LaTeX の場合}

\begin{description}
  \item[macOS・Linux] \lstinline|LATEXMKFLAG=-lualatex make|のかわりに\lstinline|WORD_FONT=hiragino-pron LATEXMKFLAG=-lualatex make|を実行する

  \item[Windows] \lstinline|make|の前に\lstinline|set WORD_FONT=hiragino-pron|を実行する
\end{description}

\section{トラブルシューティング}

\subsection{偶数頁}

編集作業をしていると、レイアウトの問題で偶数頁から開始していただくことがあります。

\lstinline|\documentclass|のオプションに\ovalbox{evenstart}をつけることで簡単にできます。

\begin{lstlisting}[language=TeX, mathescape]
\documentclass[evenstart]{word}
\end{lstlisting}

\subsubsection*{2018/04}
新しいバージョンでは\ovalbox{main\_odd.tex}と\ovalbox{main\_even.tex}にドキュメントクラスを設定し、\ovalbox{\documentclass{word}
\usepackage{txfonts}
\usepackage[multi, deluxe, bold]{otf}
\usepackage{default-style}

%\title{フッタは削除されました}
\subtitle{ヘッダの見出し}
\author{ほげ}

\begin{document}

\chapter{記事の見出し}

\section{はじめに}
......
\section{つぎに}
......
\section{おわりに}
......

\end{document}
}を
読み込むようになったので、このテンプレートを用いる場合は特に変更を行う必要はありません。

\subsection{「文 編集部」の削除}

編集部以外のメンバーが執筆する場合「文 編集部」は必要ありません。
「文 編集部」は以下のコマンドを\lstinline|\documentclass|から
\lstinline|\begin{document}|の間のどこかに書くことで消せます。

\begin{lstlisting}[language=TeX, mathescape]
\authormark{}    
\end{lstlisting}

\section{鍵の登録}

Gitサーバに鍵を登録しないと、pushできません。もしそれが原因でつまっている場合には、誰か権限を持っていそうな人に頼んで登録してもらいましょう。2016年6月現在では、
pi8027, yyu, ioriveur, shinkbr, osyoyu, chris, nymphiumが部員を登録できます。鍵が変わった場合も声をかけましょう。

\section{他の問題について}

問題があればslackの\#latexチャンネルや、編集会議で聞くと良いでしょう。

直接詳しい人にSNSで聞く場合、@\_yyu\_\footnote{\url{https://twitter.com/_yyu_}}へ投げると早い。
Lua\LaTeX に関しては@Nymphium\footnote{\url{https://twitter.com/Nymphium}}か@azuma962\footnote{\url{https://twitter.com/azuma962}}へ。
クラスファイルの全体的な質問は@hid\_alma1026\footnote{\url{https://twitter.com/hid_alma1026}}へ。

\end{document}
