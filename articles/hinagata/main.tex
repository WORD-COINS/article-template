\makeatletter
\def\input@path{{../../}}
\makeatother
 
\documentclass[../../main]{subfiles}
  
\begin{document}

% Listingsの番号再定義、この命令はdocumentがはじまってからでないとダメ
\renewcommand{\thelstlisting}{\arabic{lstlisting}}

\subtitle{ヘッダの見出し}

% 著者名の前にある「文 編集部」を削除する場合は次のコマンドをアンコメントします。
% \authormark{}

\author{情報 太郎}

\chapter{記事を執筆しよう}

\section{環境を準備する}

WORDの記事をコンパイルするにあたって、\TeX Liveといった
\TeX のツールセットは一切インストールする必要はありません。
コンパイルに必要なツールは下記です。

\begin{itemize}
  \item \href{https://git-scm.com}{Git}
  \item \href{https://www.docker.com/}{Docker}
  \item \href{https://github.com/docker/compose}{docker-compose}
\end{itemize}

\section{コンパイルする}

記事をコンパイルするため、まずは\lstinline|texfiles|ディレクトリの中で
次のコマンドを実行します。この操作は一度だけ行えば、記事を書き換えたとしても
再び行う必要はありません。

\begin{lstlisting}
$ docker-compose pull
$ docker-compose up
\end{lstlisting}

そして、記事のGitリポジトリのルートで次のコマンドを実行します。

\begin{lstlisting}
$ docker-compose pull
$ docker-compose up  
\end{lstlisting}

\section{Gitサーバにpushする}

WORDではかつて伝統あるGitoliteにより管理されていましたが、
ディスク故障によってGitのデータを失ってからBitbucketへ移行しました。

まず、WORDのGitbucketのチームである\url{https://bitbucket.org/word-coins}へ
参加していない場合は、Bitbucketのアカウントを取得して
管理者の誰かへ連絡してメンバーに加えてもらいましょう。

Bitbukectにプッシュする場合は、自分の編集を行った適当なブランチを
次のコマンドでプッシュします。

\begin{lstlisting}[mathescape]
git push origin feature/my_branch
\end{lstlisting}

プッシュを成功させた場合には、ビルドの結果がSlack\footnote{\url{https://word-ac.slack.com}}の\texttt{\#latex}チャンネルに流れます。

\section{トラブルシューティング}

\subsection{「文 編集部」の削除}

編集部以外のメンバーが執筆する場合「文 編集部」は必要ありません。
「文 編集部」は以下のコマンドを\lstinline|\documentclass|から
\lstinline|\begin{document}|の間のどこかに書くことで消せます。

\begin{lstlisting}[language=TeX, mathescape]
\authormark{}    
\end{lstlisting}

\subsection{\texttt{\textbackslash input}コマンド}

\lstinline|\input|コマンドを利用する場合、
このフォルダからの相対パスではなくリポジトリルートからの相対パスを利用してください。

\section{他の問題について}

問題があればslackの\#latexチャンネルや、編集会議で聞くと良いでしょう。

直接詳しい人にSNSで聞く場合、@\_yyu\_\footnote{\url{https://twitter.com/_yyu_}}へ投げると早い。
Lua\LaTeX に関しては@Nymphium\footnote{\url{https://twitter.com/Nymphium}}か@azuma962\footnote{\url{https://twitter.com/azuma962}}へ。
クラスファイルの全体的な質問は@hid\_alma1026\footnote{\url{https://twitter.com/hid_alma1026}}へ。

\end{document}
