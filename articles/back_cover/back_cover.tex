% ---
% 裏表紙生成用texファイル
% 同梱の `config.tex` をイジると、内容の変更ができます
% ---
\documentclass{word}
\usepackage{graphicx, framed}
\makeatletter
\renewcommand{\large}{\@setfontsize\large{30pt}{30pt}}
\renewcommand{\normalsize}{\@setfontsize\normalsize{24pt}{24pt}}
\renewcommand{\small}{\@setfontsize\small{18pt}{18pt}}
\renewcommand{\footnotesize}{\@setfontsize\footnotesize{14pt}{14pt}}
\newcommand*{\circulation}[1]{\gdef\@circulation{#1}}
\makeatother
\pagestyle{empty}
\setlength{\parindent}{0pt}
\renewcommand{\arraystretch}{1.5}
%%\defaultfontfeatures{Mapping=tex-text}
%%\defaultjfontfeatures{Mapping=tex-text}
%%\setromanfont{TeX Gyre Termes}
\newcommand*{\vstretch}[1]{\vspace*{\stretch{#1}}}

\input config.tex

\begin{document}
情報科学類誌
\begin{figure}[h]
  \centering
  \includegraphics[width=\textwidth]{wordlogo.pdf}
\end{figure}

\vstretch{3}

\makeatletter
\begin{center}
  {\large \@title}
\end{center}

\vstretch{4}

\begin{framed}
  \centering
  \begin{tabular}{ll}
    発行者 & 情報科学類長 \\
    編集長 & \@author{} \\
    制作・編集 & \shortstack[l]{%
                  \footnotesize 筑波大学情報学群 \\%
                  \footnotesize 情報科学類WORD編集部 \\%
                  \footnotesize (第三エリアC棟212号室)}
  \end{tabular}

  \vspace*{0.3em}

  {\small \@date{}\hspace{1em}初版第1刷発行\hfill{}(\@circulation{}部)}

  \vspace*{0.2em}
\end{framed}
\makeatother
\end{document}