\documentclass{word}

\usepackage[luatex]{graphicx}
\usepackage{framed}
\usepackage[svgnames,luatex]{xcolor}
\usepackage{listings}
\usepackage{fancybox}
\usepackage[%
luatex,%
hypertexnames=false%
]{hyperref}

% 参考文献
\usepackage[
backend=bibtex,%
bibstyle=numeric,%
sortcites=true,%
]{biblatex}
\addbibresource{main.bib}
\defbibheading{subbibnumbered}{%
	\section{参考文献}%
}

\usepackage{tikz}

\usepackage{pdfpages}

% CJKフォントを使いたい場合、dockerにNotoSerifCJKをインストールするよう設定した後
% 以下をコメントアウト
%\newfontfamily{\cjkfont}{Noto Serif CJK}

\begin{document}

% Listingsの番号再定義です。
% この命令はdocumentがはじまってからでないとダメなのでここにあります。
\renewcommand{\thelstlisting}{\arabic{lstlisting}}
% 表紙
\frontmatter

\cover{./articles/cover}
\newpage
% 目次(チャプターレベルしか目次に表示しない)
\setcounter{tocdepth}{0}
\thispagestyle{empty}
\tableofcontents
\newpage

\mainmatter
\pagestyle{fancy}
% 記事(サンプル)
\article{./articles/hinagata}
% 記事(markdownサンプル)
\article{./articles/hinagata-markdown}
% 編集後記用プレースホルダ
\phantomsection
\addcontentsline{toc}{chapter}{編集後記}
\newpage
% 裏表紙
\article{./articles/back_cover}

\end{document}
